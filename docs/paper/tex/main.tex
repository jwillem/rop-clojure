\documentclass[10pt,journal,compsoc]{IEEEtran}

\usepackage[ngerman]{babel}
\usepackage[utf8]{inputenc}
\usepackage[T1]{fontenc}
\usepackage{graphicx}
\usepackage{listings}

\graphicspath{{images/}}

% correct bad hyphenation here
\hyphenation{}

\begin{document}
  \title{Railway-Oriented-Programming in Clojure -- Funktionales Error-Handling}
  \author{Jan-Philipp~Willem,~Fakultät für Informatik, Hochschule Mannheim}
  \markboth{Railway-Oriented-Programming in Clojure -- Funktionales Error-Handling}{}

  \IEEEtitleabstractindextext{%
    \begin{abstract}
    \\
    ~\\
    \textbf{Abstract}---
    \end{abstract}
  }
  \maketitle

  \IEEEraisesectionheading{
    \section{Einleitung}\label{sec:einleitung}
  }
  \IEEEPARstart{S}{eit} es losging.
  \subsection{"`Happy-path"'}
  \subsection{"`Error-by-design"'}

  \section{Railway-Oriented-Programming}
  \section{User-Service mithilfe von ROP}
  \section{Fazit}
  \begin{thebibliography}{1}

    \bibitem{railwayWlaschin}
      Scott Wlaschin, \emph{Railway-Oriented-Programming -- a functional approach to error handling}, https://fsharpforfunandprofit.com/rop/
  \end{thebibliography}
\end{document}
