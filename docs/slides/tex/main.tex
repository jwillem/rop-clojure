\documentclass[compress]{beamer}
\usepackage[ngerman]{babel}
\usepackage{graphicx}
\usepackage{subfiles}
\usepackage{listings}

\graphicspath{{images/}}

% \setbeameroption{show notes}

\usetheme[noflama]{custom}

\title{Railway-Oriented-Programming \break in Clojure}
\subtitle{Funktionales Error-Handling}
\author{Jan-Philipp Willem}
\institute{Fakultät für Informatik\\Hochschule Mannheim}
\date{EFP, SS2017}

\begin{document}

% \begin{frame}[noframenumbering,plain]
% \end{frame}

% TODO slide background
\maketitle

\section*{Gliederung}
\begin{frame}[noframenumbering,plain]{Gliederung}
  \tableofcontents[hideallsubsections]
\end{frame}

\section{Railway-Oriented-Programming}
  \begin{frame}{"`Happy-Path"'}
  \setcounter{framenumber}{1}
    \begin{columns}[c]
    \column{.5\textwidth}
      \begin{itemize}
        \item first
        \item<2-> second
        \item<3-> third
        \item<4-> fourth
      \end{itemize}
    \column{.5\textwidth}
    % \includegraphics[width=\textwidth]{parallel.pdf}
    \end{columns}
  \end{frame}

  \note[itemize]{
      \item 
  }
\end{document}
